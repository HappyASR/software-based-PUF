% \chapter{\chapterThree}
% \label{chp:3}
\section{Problem Statement}\label{ch:problem}


Since introduced by Guajardo and Holcomb in 2007, there have been many innovations in SRAM PUF field. A simple patent search using patents.google.com with query 'sram; puf' results in 546 results \cite{google_patents}. The number of articles in \seqsplit{scholar.google.com} also exhibit a high occurrences, shown by a total of 2,120 articles (citations and patents are not included) \cite{google_scholar}.
Even though these facts indicate a promising future for this concept, one also should notice that current state-of-the-art in this field mostly consists of one-off prototypes or specific proprietary implementations.
To get an SRAM PUF product from the market, one has to order a specific request from a company. For example, Intrinsic-ID, one of the main leaders in SRAM PUF technology, has a software-based solution which able to generate unique keys and identities for nearly all microcontrollers without a need for security-dedicated silicon \cite{broadkey}. Even though this solution exists and seems easy to use, unfortunately, they do not say specifically how much will it cost to use this solution.
They also have another solution for SRAM PUF which is focused on hardware IP (and supporting software/firmware) to enable designers to implement PUFs within their design.
This solution has a high possibility to obstruct a small company or a single user to use their solution since usually this type of product are intended to use with a specific contract.
Similar to the software-based solution they offer, they also do not put the explicit price to use this product. An example of a product that uses this solution is FPGA Microsemi Polarfire \cite{polarfire}.

The SRAM PUF field lacks an Arduino, Linux, or GCC type of open reference implementation. A quick lookup in Github shows that there is no extensive open source project related to SRAM PUF there. There are projects corresponding to PUF concepts, but most of them also only delve into a simulation.
The communities seem to have not established a wide agreement on which approach yields the strongest security properties.

An additional issue that we would like to address is SRAM PUF's application. As mentioned in Section 1.1, the importance of securing key and user's data is getting higher, especially with the introduction of self-sovereign identity. There are already many SRAM PUF applications published, but sadly, there is no working project that tries to integrate SRAM PUF in self-sovereign identity concept. Most PUF applications are designed for authentication \cite{Tuyls2007} \cite{delvaux} \cite{Suh:2007:PUF:1278480.1278484} \cite{10.1007/978-3-642-04474-8_22} \cite{10.1007/978-3-642-10838-9_22} \cite{10.1007/978-3-319-29078-2_5}
and generating cryptographic keys \cite{Suh:2007:PUF:1278480.1278484} \cite{10.1007/978-3-642-33027-8_18}.

Based on these facts, we believe the next challenge for this field is to discover a common approach. The field needs to move beyond isolated single-person projects and single-company approaches towards a mature and sharing ecosystem. The field SRAM PUF requires a single implementation which is continuously improved upon for many years to come and is supported by the majority of the academics and commercial parties.
Furthermore, we also try to initiate an integration between PUF and self-sovereign identity by providing a scheme to protect user's data and key. This project will be useful in the process of self-sovereign identity development.

To understand our intention in this thesis better, this thesis' problem statement is presented here. The problem statement of this thesis is:

\begin{adjustwidth}{1cm}{1cm}
		\textit{\problemStatement}
		% \textit{How to develop an open source secure data protection and key storage scheme using off-the-shelf SRAM component based on SRAM PUF technology?}
    % \textit{How to develop a secure software-only SRAM PUF-based data protection and key storage scheme using off-the-shelf SRAM while providing a mature and sharing ecosystem for continuous SRAM PUF development?}
\end{adjustwidth}

Derived from the problem statement, there are two goals defined in this thesis. The first goal is to devise a secure data and key storage scheme based on SRAM PUF technology. The data and the key protected by the scheme has to be safe even though the PUF device is lost. Moreover, the scheme should work using off-the-shelf SRAM. This sub-goal leads us to another question, can we build SRAM PUF using off-the-shelf SRAM? If it is possible, what characteristics need to be fulfilled by off-the-shelf SRAM to be eligible as a PUF candidate?
In addition, the constructed SRAM PUF has to work without any hardware design, or in other words, software-based construction. The data protection and key storage functions inside the scheme will be helpful in addressing the problem of self-sovereign identity and keeping the secret key.
The next goal is to create a sharing ecosystem for the evolution of our data protection and key storage scheme. The ecosystem should be easily accessed and understood to encourage the academics and commercial parties to use and develop the ecosystem together. The easiest step to achieve this goal is by making our thesis as an open source project.

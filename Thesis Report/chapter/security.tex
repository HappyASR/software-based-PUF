\chapter{Introduction to Security}

\section{Secure System Necessity}
How valuable is our data? How much would company for accessing those information? These questions might be silly but if we consider that there are many companies which thrived using our data (e.g. Facebook, Twitter, and Uber), we should reconsider how much should we value our data. I know these data if we value each element might not worth much, most will worth significant when these data is combined together to bring a more insight information. But what about the sensitive data which its value can worth millions of dollars, such as the bitcoin key? In \cite{bitcoin}, the highest bitcoin address is worth 1.4B US\$ and there are 513,562 addresses which has value more than 10000 US\$. Due to their high values, these data, the key to these addresses, must be protected.

Before going further, we should understand first what kind of attacks possibly affecting these data. Similar like in the physical world, in the digital world, adversaries' intention can be either mischievious, non-malicious or accidental. Some example of mischievious activities are stealing information and modifying the data. Accidental can happen due to human error. These three types of adversaries' intention can lead to a significant number of attacks and threats.

Now, two questions are arised. Is there are a way that guarantee 100\% of these data protection? Is there any bullet-proof secure system? Unfortunately, there is no such thing as a 100\% secure system. Fortunately, there are ways to design a system to be as secure as possible in a limited scope, usually defined as secure 'from who' and 'from what'. According to \cite{Pfleeger}, computer security is ``the protection of the items you value, called the assets of a computer or a computer system." In the scope of data mentioned before, the assets are the bitcoin keys and addresses.

To help defining a secure system, common security requirements are mentioned.
According to \cite{cryptography_decrypted}, there are four elements on common security, which are:
\begin{itemize}
  \item \textit{Confidentiality}: a piece of information should be accesible only to an authorized users. For example, an encrypted data can only be decrypted by the secret key owner.
  \item \textit{Authentication}: assurance of the sender of a message, date of origin, data content, time sent, data information, etc. are correctly identified.
  \item \textit{Integrity}: any assets can only be modified by an authorized subjects. For example, data should be keep intact during transmission
  \item \textit{Non-repudiation}: a subject should be prevented from denying previous actions. For example, a sender cannot deny the data which it sent.
\end{itemize}

\section{Cryptography}
One way to achieve these four security requirements is by using cryptography. In traditional definition, cryptography can be defined as the art of writing or solving codes \cite{Oxford_dictionary}. But this definition is inaccurate to use nowadays because instead of depending on creativity and personal skill when constructing or breaking codes, the modern cryptography focus their definition using science and mathematics. According to \cite{modern_cryptography}, modern cryptography can be defined as ``the scientific study of techniques for securing digital information, transactions, and distributed computations." The algorithm which use cryptography as their main point is called cryptographic algorithm.

Since the birth of cryptography, its main concerned is usually related on securing communication which can be achieved by constructing \textit{ciphers} to provide secret communication between parties involved. The construction of ciphers to ensure only authorized parties also can be called as encryption schemes.
There are two types of cryptographic algorithm, symmetric and asymmetric algorithm. Symmetric, also known as private key encryption or private key cryptography, requires the same key for encryption and decryption. Meanwhile in asymmetric algorithm (can be referred as public key encryption or public key cryptography), there are two keys utilized; private key and public key. Public key is utilized for encryption and private key is used for decryption. One of the main advantage of symmetric encryption over asymmetric encryption is it requires less computational power which make it suitable to use in embedded devices.
Further explanation on symmetric encryption algorithm will be provided in the next chapter.
